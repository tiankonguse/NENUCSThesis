\chapter{样本测试}
\label{cha:samples}
这一章我将对各种科技论文中常见的格式进行展示。其中大量的示例都来自于
ThuThesis \cite{thuthesis-website} 项目,在此对他们的工作表示感谢

\section{字体测试}
\subsection{中文测试}
\begin{itemize}
\item 宋体:{\song 我能吞下玻璃而不伤身体}
\item 黑体:{\hei 我能吞下玻璃而不伤身体}
\item 楷体:{\kai 我能吞下玻璃而不伤身体}
\end{itemize}

\subsection{形状测试}
\begin{itemize}
\item 直立:\textup{ 我能吞下玻璃而不伤身体 ABCDEFG abcdefg}
\item 小体大写:\textsc{ 我能吞下玻璃而不伤身体 ABCDEFG abcdefg}
\item 意大利斜体:\textit{ 我能吞下玻璃而不伤身体 ABCDEFG abcdefg}
\item Slanted斜体:\textsl{ 我能吞下玻璃而不伤身体 ABCDEFG abcdefg}
\item 加粗:\textbf{ 我能吞下玻璃而不伤身体 ABCDEFG abcdefg}
\end{itemize}


\section{公式样本测试}
\label{sec:sample for formula}
在第~\ref{cha:intro} 章中我提到 \LaTeX 的公式非常漂亮~
(\ref{subsec:best-formula}) ,那么我们来看看效果到底如何。贝叶斯公式:
\begin{equation}
\label{equ:chap3:bayes}
p(y|\mathbf{x}) = \frac{p(\mathbf{x},y)}{p(\mathbf{x})}=
\frac{p(\mathbf{x}|y)p(y)}{p(\mathbf{x})} 
\end{equation}

论文里面公式越多,\TeX{} 就越 happy。再看一个 \textsf{amsmath} 的例子,也来自ThuThesis \cite{thuthesis-website}:
\begin{equation}
\label{detK2}
\det\mathbf{K}(t=1,t_1,\dots,t_n)=\sum_{I\in\mathbf{n}}(-1)^{\envert{I}}
\prod_{i\in I}t_i\prod_{j\in I}(D_j+\lambda_jt_j)\det\mathbf{A}
^{(\lambda)}(\overline{I}|\overline{I})=0.
\end{equation} 

还有矩阵 
\begin{equation}
A =
\begin{pmatrix}                % 使用小括號
  t_{11} & t_{12} & t_{13} \\
  t_{21} & t_{22} & t_{23} \\
  t_{31} & t_{32} & t_{33}
\end{pmatrix}
\end{equation} 


\section{表格测试}
\subsection{基本表格}
\label{sec:basictable}

模板中关于表格的宏包有三个: \textsf{booktabs}、\textsf{array} 和
\textsf{longtabular},命令有一个 \verb|\hlinewd|。三线表可以用 \textsf{booktabs}
提供的 \verb|\toprule|、\verb|\midrule| 和 \verb|\bottomrule|。它们与
\textsf{longtable} 能很好的配合使用。如果表格比较简单的话可以直接用命令
\verb|hlinewd{xpt}| 控制。
\begin{table}[H]
  \centering
  \begin{minipage}[t]{0.8\linewidth} % 如果想在表格中使用脚注,minipage是个不错的办法
  \caption[模板文件]{模板文件。如果表格的标题很长,那么在表格索引中就会很不美
    观,所以要像 chapter 那样在前面用中括号写一个简短的标题。这个标题会出现在索
    引中。}
  \label{tab:template-files}
    \begin{tabular*}{\linewidth}{lp{10cm}}
      \toprule[1.5pt]
      {\hei 文件名} & {\hei 描述} \\\midrule[1pt]
      nenuthesis.cls    & 模板类文件\\
      template.tex      & 样例tex文件 \\
      pages/cover.tex   & 封面,包含英文封面和中文封面\\
      pages/abse.tex    & 英文摘要 \\
      pages/absc.tex    & 中文摘要 \\
      pages/intro.tex   & 绪论,即第一章\\
      pages/relateworks.tex     & 相关工作,即第二章 \\
      pages/chap03.tex  & 第三章,一大堆样例 \\
      figures/      & 图片目录 \\
      ref/refs      & 参考文献 \\ 
      \bottomrule[1.5pt]
    \end{tabular*}
  \end{minipage}
\end{table}
首先来看一个最简单的表格。表 \ref{tab:template-files} 列举了本模板主要文件及
其功能。请大家注意三线表中各条线对应的命令。这个例子还展示了如何在表格中正确
使用脚注。

由于 \LaTeX{} 本身不支持在表格中使用 \verb|\footnote|,所以我们不得不将表格放在
小页中,而且最好将表格的宽度设置为小页的宽度,这样脚注看起来才更美观。

\subsection{复杂表格}
\label{sec:complicatedtable}

我们经常会在表格下方标注数据来源,或者对表格里面的条目进行解释。前面的脚注是一种。

\begin{table}[H]
  \centering
  \caption{复杂表格示例 1}
  \label{tab:tabexamp1}
  \begin{minipage}[t]{0.8\textwidth} 
    \begin{tabularx}{\linewidth}{|l|X|X|X|X|}
      \hline
 \multirow{2}*{\backslashbox{x}{y}}  & \multicolumn{2}{c|}{First Half} & \multicolumn{2}{c|}{Second Half}\\\cline{2-5}
      & 1st Qtr &2nd Qtr&3rd Qtr&4th Qtr \\ \hline
      East$^{*}$ &   20.4&   27.4&   90&     20.4 \\
      West$^{**}$ &   30.6 &   38.6 &   34.6 &  31.6 \\ \hline
    \end{tabularx} \\[2pt]
    \footnotesize 注:数据来源《\nenuthesis{} 使用手册》。\\
    *:东部\\
    **:西部
  \end{minipage}
\end{table}

此外,表~\ref{tab:tabexamp1} 同时还演示了另外两个功能:1)通过 \textsf{tabularx} 的
 \texttt{|X|} 扩展实现表格自动放大;2)通过命令 \verb|\backslashbox| 在表头部分
插入反斜线。

\subsection{超级长表格}
\textsf{supertabular} 
宏包,模板对 \textsf{longtable} 进行了相应的设置,所以用起来可能简单一些。
表~\ref{tab:performance} 就是 \textsf{longtable} 的简单示例。
\begin{longtable}[c]{c*{6}{r}}
\caption{实验数据}\label{tab:performance}\\
\toprule[1.5pt]
 测试程序 & \multicolumn{1}{c}{正常运行} & \multicolumn{1}{c}{同步} & \multicolumn{1}{c}{检查点} & \multicolumn{1}{c}{卷回恢复}
& \multicolumn{1}{c}{进程迁移} & \multicolumn{1}{c}{检查点} \\
& \multicolumn{1}{c}{时间 (s)}& \multicolumn{1}{c}{时间 (s)}&
\multicolumn{1}{c}{时间 (s)}& \multicolumn{1}{c}{时间 (s)}& \multicolumn{1}{c}{
  时间 (s)}&  文件(KB)\\\midrule[1pt]
\endfirsthead
\multicolumn{7}{c}{续表~\thetable\hskip1em 实验数据}\\
\toprule[1.5pt]
 测试程序 & \multicolumn{1}{c}{正常运行} & \multicolumn{1}{c}{同步} & \multicolumn{1}{c}{检查点} & \multicolumn{1}{c}{卷回恢复}
& \multicolumn{1}{c}{进程迁移} & \multicolumn{1}{c}{检查点} \\
& \multicolumn{1}{c}{时间 (s)}& \multicolumn{1}{c}{时间 (s)}&
\multicolumn{1}{c}{时间 (s)}& \multicolumn{1}{c}{时间 (s)}& \multicolumn{1}{c}{
  时间 (s)}&  文件(KB)\\\midrule[1pt]
\endhead
\hline
\multicolumn{7}{r}{续下页}
\endfoot
\endlastfoot
CG.A.2 & 23.05 & 0.002 & 0.116 & 0.035 & 0.589 & 32491 \\
CG.A.4 & 15.06 & 0.003 & 0.067 & 0.021 & 0.351 & 18211 \\
CG.A.8 & 13.38 & 0.004 & 0.072 & 0.023 & 0.210 & 9890 \\
CG.B.2 & 867.45 & 0.002 & 0.864 & 0.232 & 3.256 & 228562 \\
CG.B.4 & 501.61 & 0.003 & 0.438 & 0.136 & 2.075 & 123862 \\
CG.B.8 & 384.65 & 0.004 & 0.457 & 0.108 & 1.235 & 63777 \\
MG.A.2 & 112.27 & 0.002 & 0.846 & 0.237 & 3.930 & 236473 \\
MG.A.4 & 59.84 & 0.003 & 0.442 & 0.128 & 2.070 & 123875 \\
MG.A.8 & 31.38 & 0.003 & 0.476 & 0.114 & 1.041 & 60627 \\
MG.B.2 & 526.28 & 0.002 & 0.821 & 0.238 & 4.176 & 236635 \\
MG.B.4 & 280.11 & 0.003 & 0.432 & 0.130 & 1.706 & 123793 \\
MG.B.8 & 148.29 & 0.003 & 0.442 & 0.116 & 0.893 & 60600 \\
LU.A.2 & 2116.54 & 0.002 & 0.110 & 0.030 & 0.532 & 28754 \\
LU.A.4 & 1102.50 & 0.002 & 0.069 & 0.017 & 0.255 & 14915 \\
LU.A.8 & 574.47 & 0.003 & 0.067 & 0.016 & 0.192 & 8655 \\
LU.B.2 & 9712.87 & 0.002 & 0.357 & 0.104 & 1.734 & 101975 \\
LU.B.4 & 4757.80 & 0.003 & 0.190 & 0.056 & 0.808 & 53522 \\
LU.B.8 & 2444.05 & 0.004 & 0.222 & 0.057 & 0.548 & 30134 \\
EP.A.2 & 123.81 & 0.002 & 0.010 & 0.003 & 0.074 & 1834 \\
EP.A.4 & 61.92 & 0.003 & 0.011 & 0.004 & 0.073 & 1743 \\
EP.A.8 & 31.06 & 0.004 & 0.017 & 0.005 & 0.073 & 1661 \\
EP.B.2 & 495.49 & 0.001 & 0.009 & 0.003 & 0.196 & 2011 \\
EP.B.4 & 247.69 & 0.002 & 0.012 & 0.004 & 0.122 & 1663 \\
EP.B.8 & 126.74 & 0.003 & 0.017 & 0.005 & 0.083 & 1656 \\
\bottomrule[1.5pt]
\end{longtable}

\section{定理环境}
\label{sec:theorem}

给大家演示一下各种和证明有关的环境:
\begin{assumption}
待月西厢下,迎风户半开;隔墙花影动,疑是玉人来。
\begin{eqnarray}
  \label{eq:eqnxmp}
  c & = & a^2 - b^2\\
    & = & (a+b)(a-b)
\end{eqnarray}
\end{assumption}

千辛万苦,历尽艰难,得有今日。然相从数千里,未曾哀戚。今将渡江,方图百年欢笑,如
何反起悲伤?(引自《杜十娘怒沉百宝箱》)

\begin{definition}
子曰:「道千乘之国,敬事而信,节用而爱人,使民以时。」
\end{definition}

千古第一定义!问世间、情为何物,只教生死相许?天南地北双飞客,老翅几回寒暑。欢乐趣,离别苦,就中更有痴儿女。
君应有语,渺万里层云,千山暮雪,只影向谁去?

横汾路,寂寞当年箫鼓,荒烟依旧平楚。招魂楚些何嗟及,山鬼暗谛风雨。天也妒,未信与,莺儿燕子俱黄土。
千秋万古,为留待骚人,狂歌痛饮,来访雁丘处。

\begin{proposition}
 曾子曰:「吾日三省吾身 \pozhehao 为人谋而不忠乎?与朋友交而不信乎?传不习乎?」
\end{proposition}

多么凄美的命题啊!其日牛马嘶,新妇入青庐,奄奄黄昏后,寂寂人定初,我命绝今日,
魂去尸长留,揽裙脱丝履,举身赴清池,府吏闻此事,心知长别离,徘徊庭树下,自挂东南
枝。

\begin{remark}
天不言自高,水不言自流。
\begin{gather*}
\begin{split} 
\varphi(x,z)
&=z-\gamma_{10}x-\gamma_{mn}x^mz^n\\
&=z-Mr^{-1}x-Mr^{-(m+n)}x^mz^n
\end{split}\\[6pt]
\begin{align} \zeta^0&=(\xi^0)^2,\\
\zeta^1 &=\xi^0\xi^1,\\
\zeta^2 &=(\xi^1)^2,
\end{align}
\end{gather*}
\end{remark}

天尊地卑,乾坤定矣。卑高以陈,贵贱位矣。 动静有常,刚柔断矣。方以类聚,物以群分,
吉凶生矣。在天成象,在地成形,变化见矣。鼓之以雷霆,润之以风雨,日月运行,一寒一
暑,乾道成男,坤道成女。乾知大始,坤作成物。乾以易知,坤以简能。易则易知,简则易
从。易知则有亲,易从则有功。有亲则可久,有功则可大。可久则贤人之德,可大则贤人之
业。易简,而天下矣之理矣;天下之理得,而成位乎其中矣。

\begin{axiom}
两点间直线段距离最短。  
\begin{align}
x&\equiv y+1\pmod{m^2}\\
x&\equiv y+1\mod{m^2}\\
x&\equiv y+1\pod{m^2}
\end{align}
\end{axiom}

《彖曰》:大哉乾元,万物资始,乃统天。云行雨施,品物流形。大明始终,六位时成,时
乘六龙以御天。乾道变化,各正性命,保合大和,乃利贞。首出庶物,万国咸宁。

《象曰》:天行健,君子以自强不息。潜龙勿用,阳在下也。见龙再田,德施普也。终日乾
乾,反复道也。或跃在渊,进无咎也。飞龙在天,大人造也。亢龙有悔,盈不可久也。用九,
天德不可为首也。   

\begin{lemma}
《猫和老鼠》是我最爱看的动画片。
\begin{multline*}%\tag*{[a]} % 这个不出现在索引中
\int_a^b\biggl\{\int_a^b[f(x)^2g(y)^2+f(y)^2g(x)^2]
 -2f(x)g(x)f(y)g(y)\,dx\biggr\}\,dy \\
 =\int_a^b\biggl\{g(y)^2\int_a^bf^2+f(y)^2
  \int_a^b g^2-2f(y)g(y)\int_a^b fg\biggr\}\,dy
\end{multline*}
\end{lemma}

行行重行行,与君生别离。相去万余里,各在天一涯。道路阻且长,会面安可知。胡马依北
风,越鸟巢南枝。相去日已远,衣带日已缓。浮云蔽白日,游子不顾返。思君令人老,岁月
忽已晚。  弃捐勿复道,努力加餐饭。

\begin{theorem}\label{the:theorem1}
犯我强汉者,虽远必诛\hfill \pozhehao 陈汤(汉)
\end{theorem}
\begin{subequations}
\begin{align}
y & = 1 \\
y & = 0
\end{align}
\end{subequations}
道可道,非常道。名可名,非常名。无名天地之始;有名万物之母。故常无,欲以观其妙;
常有,欲以观其徼。此两者,同出而异名,同谓之玄。玄之又玄,众妙之门。上善若水。水
善利万物而不争,处众人之所恶,故几于道。曲则全,枉则直,洼则盈,敝则新,少则多,
多则惑。人法地,地法天,天法道,道法自然。知人者智,自知者明。胜人者有力,自胜
者强。知足者富。强行者有志。不失其所者久。死而不亡者寿。

\begin{proof}
燕赵古称多感慨悲歌之士。董生举进士,连不得志于有司,怀抱利器,郁郁适兹土,吾
知其必有合也。董生勉乎哉?

夫以子之不遇时,苟慕义强仁者,皆爱惜焉,矧燕、赵之士出乎其性者哉!然吾尝闻
风俗与化移易,吾恶知其今不异于古所云邪?聊以吾子之行卜之也。董生勉乎哉?

吾因子有所感矣。为我吊望诸君之墓,而观于其市,复有昔时屠狗者乎?为我谢
曰:“明天子在上,可以出而仕矣!” \hfill\pozhehao 韩愈《送董邵南序》
\end{proof}

\begin{corollary}
  四川话配音的《猫和老鼠》是世界上最好看最好听最有趣的动画片。
\begin{alignat}{3}
V_i & =v_i - q_i v_j, & \qquad X_i & = x_i - q_i x_j,
 & \qquad U_i & = u_i,
 \qquad \text{for $i\ne j$;}\label{eq:B}\\
V_j & = v_j, & \qquad X_j & = x_j,
  & \qquad U_j & u_j + \sum_{i\ne j} q_i u_i.
\end{alignat}
\end{corollary}

迢迢牵牛星,皎皎河汉女。
纤纤擢素手,札札弄机杼。
终日不成章,泣涕零如雨。
河汉清且浅,相去复几许。
盈盈一水间,脉脉不得语。

\begin{example}
  大家来看这个例子。
\begin{equation}
\label{ktc}
\left\{\begin{array}{l}
\nabla f({\mbox{\boldmath $x$}}^*)-\sum\limits_{j=1}^p\lambda_j\nabla g_j({\mbox{\boldmath $x$}}^*)=0\\[0.3cm]
\lambda_jg_j({\mbox{\boldmath $x$}}^*)=0,\quad j=1,2,\cdots,p\\[0.2cm]
\lambda_j\ge 0,\quad j=1,2,\cdots,p.
\end{array}\right.
\end{equation}
\end{example}

\begin{exercise}
  清列出 Andrew S. Tanenbaum 和 W. Richard Stevens 的所有著作。
\end{exercise}

\begin{conjecture} \textit{Poincare Conjecture} If in a closed three-dimensional
  space, any closed curves can shrink to a point continuously, this space can be
  deformed to a sphere.
\end{conjecture}

\begin{problem}
 回答还是不回答,是个问题。 
\end{problem}

如何引用定理~\ref{the:theorem1} 呢?加上 \verb|label| 使用 \verb|ref| 即可。妾发
初覆额,折花门前剧。郎骑竹马来,绕床弄青梅。同居长干里,两小无嫌猜。 十四为君妇,
羞颜未尝开。低头向暗壁,千唤不一回。十五始展眉,愿同尘与灰。常存抱柱信,岂上望夫
台。 十六君远行,瞿塘滟滪堆。五月不可触,猿声天上哀。门前迟行迹,一一生绿苔。苔深
不能扫,落叶秋风早。八月蝴蝶来,双飞西园草。感此伤妾心,坐愁红颜老。


\section{绘图与插图样本}
\label{sec:sample for draw and graphs}

\subsection{绘图}
\label{sec:draw}
本人完全不懂如何使用宏包绘图,因此如果有需要请自行加载对应绘图包(如 \textsf{pstricks,pgf} 等)

\subsection{插图}
\label{sec:graphs}
我常用的做图工具有Dia, Yed, GIMP, Inkscape, ImageMagicK graphviz。其中前四个为 GUI 绘图软体(噢,我还顺便测试了列表环境):
\begin{itemize}
\item DIA:绘制各种图表的工具,包括UML、ER..
\item Yed:一个Java Swing轻量级绘图工具。非自由软体。
\item GIMP:类似PhotoShop的强大位图处理程序。
\item Inkscape:可能是自由软体中最好用的矢量图形绘制程序
\end{itemize}

graphviz 和 ImageMagicK 是命令行程序,很适合自动化图形处理:
\begin{itemize}
\item Graphviz:Graph drawing addresses the problem of visualizing structural information
by constructing geometric representations of abstract graphs and networks
\item ImageMagicK:ImageMagick is a software suite to create, edit, and compose bitmap images.
\end{itemize}

强烈推荐《\LaTeXe 插图指南》!关于子图形的使用细节请参看 \textsf{subfig} 的说明文档。

\subsubsection{一个图形}
\label{sec:onefig}
一般图形都是处在浮动环境中。之所以称为浮动是指最终排版效果图形的位置不一定与源文
件中的位置对应\footnote{This is not a bug, but a feature of \LaTeX!},这也是刚使
用 \LaTeX{} 同学可能遇到的问题。如果要强制固定浮动图形的位置,请使用 \textsf{float} 宏包,
它提供了 \texttt{[H]} 参数,比如图~\ref{fig:xfig1}。
\begin{figure}[H] % use float package if you want it here
  \centering
  \includegraphics[width=100mm]{test_diagram1}
  \caption{UML示例:继承关系}
  \label{fig:xfig1}
\end{figure}

大学之道,在明明德,在亲民,在止于至善。知止而后有定;定而后能静;静而后能安;安
而后能虑;虑而后能得。物有本末,事有终始。知所先后,则近道矣。古之欲明明德于天
下者,先治其国;欲治其国者,先齐其家;欲齐其家者,先修其身;欲修其身者,先正其心;

\hfill \pozhehao《大学》

如果要把编号的两个图形并排,那么小页就非常有用了:
\begin{figure}[H]
\begin{minipage}{0.48\textwidth}
  \centering
  \includegraphics[height=5cm]{test_diagram2}
  \caption{并排第一个图}
  \label{fig:parallel1}
\end{minipage}\hfill
\begin{minipage}{0.48\textwidth}
  \centering
  \includegraphics[height=5cm]{test_diagram2}
  \caption{并排第二个图}
  \label{fig:parallel2}
\end{minipage}
\end{figure}

李氏子蟠,年十七,好古文、六艺,经传皆通习之,不拘於时,学於余。余嘉其能行古
道,作师说以贻之。

\subsubsection{多个图形}
\label{sec:multifig}

如果多个图形相互独立,并不共用一个图形计数器,那么用 \verb|minipage| 或者
\verb|parbox| 就可以。否则,请参看图~\ref{fig:big1},它包含两个小图,分别是图~\ref{fig:subfig1} 
和图~\ref{fig:subfig2}。推荐使用 \verb|\subfloat|,不要再用
\verb|\subfigure| 和 \verb|\subtable|。
\begin{figure}[H]
  \centering%
  \subfloat[第一个小图形]{%
    \label{fig:subfig1}
    \includegraphics[height=6cm]{test_diagram2}}\hspace{4em}%
  \subfloat[第二个小图形。如果标题很长的话,它会自动换行,这个 caption 就是这样的例子]{%
    \label{fig:subfig2}
    \includegraphics[height=6cm]{test_diagram2}}
  \caption{包含子图形的大图形}
  \label{fig:big1}
\end{figure}


